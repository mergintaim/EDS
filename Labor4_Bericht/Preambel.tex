%%%%%%%%%%%%%%%%%%%%%%% Schrift und Sprache %%%%%%%%%%%%%%%%%%%%%%%%%%%%%%
\usepackage[ngerman]{babel}							% Sprachregeln für Deutsch
\usepackage[utf8]{inputenc}							% für Zeicheneingabe
\usepackage[T1]{fontenc}							% für Umlaute
\usepackage{lmodern}								% deutsche Trennungsregeln usw
%\usepackage{mathptmx}								% für Times New Roman
\usepackage{eurosym}								% für Euro Zeichen

%%%%%%%%%%%%%%%%%%%%%%% 	Seitenlayout	%%%%%%%%%%%%%%%%%%%%%%%%%%%%%%
%\usepackage[onehalfspacing]{setspace}				% 1,5 Zeilenabstand
\usepackage{setspace}								% für Zeilenabstand
\usepackage{scrlayer-scrpage}						% Für Kopf und Fußzeile
\pagestyle{headings}								% Seitenstil festlegen (z.B. Kopfzeile)
\clearpairofpagestyles								% Seitenstil löschen
\usepackage{float}									% um floats festen Platz zuweisen zu können (mit H)

%%%%%%%%%%%%%%%%%%%%%%% 	Tabellen		%%%%%%%%%%%%%%%%%%%%%%%%%%%%%%
\usepackage{multicol}								% Tabellenspalten verbinden
\usepackage{multirow}								% Tabellenzeilen verbinden
\usepackage{colortbl}								% Farben in Tabellen
\usepackage{tabularx}								% für Tabellenbreite
\usepackage{diagbox}								% für Schrägstriche in Tabellen

%%%%%%%%%%%%%%%%%%%%%%% 	Auflistung		%%%%%%%%%%%%%%%%%%%%%%%%%%%%%%
\usepackage{enumerate}								%
\usepackage{enumitem}								% für Listen
%\usepackage{outlines}								% Auflistung mit Stickpunkten (besser für Unterpunkte)

%%%%%%%%%%%%%%%%%%%%%%%		Mathematik		%%%%%%%%%%%%%%%%%%%%%%%%%%%%%%
\usepackage{siunitx}								% Si-Einheiten
\DeclareSIUnit{\belmilliwatt}{Bm}					% für Einheit dBm
\DeclareSIUnit{\dBm}{\deci\belmilliwatt}			% Einheit dBm
\sisetup{locale=DE}									% für Komma statt Punkt bei Zahlen
\usepackage{amsmath}								% Mathematik Umgebungen
\usepackage{amssymb}								% Mathematische Symbole und Zeichen
\usepackage{bm}										% Fette Schrift in Mathe Umgebung
\usepackage{amstext}								% Text in Matheumgebung
\usepackage{amsfonts}								% überflüssig, da in amssymb enthalten
\usepackage{empheq}									% Boxen z.B. mit align
\usepackage{mathtools}								% für senkrechtes "=" in newcommand
\usepackage{accents}								% für Akzente in Mathe "entspricht" in newcommand

%%%%%%%%%%%%%%%%%%%%%%%		Elektronik		%%%%%%%%%%%%%%%%%%%%%%%%%%%%%%
\usepackage{trfsigns}								% Transformationszeichen (z.B. Laplace-Zeit)

%%%%%%%%%%%%%%%%%%%%%%% 	Grafiken		%%%%%%%%%%%%%%%%%%%%%%%%%%%%%%
\usepackage{float}									% für Positionen von Bildern
\usepackage{wrapfig}								% Grafiken können neben Text stehen
\usepackage{color}									% Farben Paket für Zeichnungen
\usepackage{graphicx}								% Bilder können inkludiert werden
%\graphicspath{{E:/Studium/4_Semester}} 			% Pfad von dem Ordner von Bildern

%%%%%%%%%%%%%%%%%%%%%%% 	Zeichnungen		%%%%%%%%%%%%%%%%%%%%%%%%%%%%%%
\usepackage{pgfplots}          						% Diagramme erstellen
\pgfplotsset{compat=1.16}       					% Compatibilitätsmodus

\usepackage[siunitx,EFvoltages,european]{circuitikz}	% Schaltpläne

%%%%%%%%%%%%%%%%%%%%%%% 	Diagramme		%%%%%%%%%%%%%%%%%%%%%%%%%%%%%%
\usepackage{tikz,pgfplots}
\usetikzlibrary{math, calc, automata, positioning, arrows, intersections, shapes.geometric}

%\tikzset{
%	->, % makes the edges directed
	%		>=stealth’, % makes the arrow heads bold
	%	node distance=2.5cm, % specifies the minimum distance between two nodes. Change if necessary.
	%	every state/.style={thick, fill=gray!10}, % sets the properties for each ’state’ node
	%	initial text=$reset$, % sets the text that appears on the start arrow
%}

%%%%%%%%%%%%%%%%%%%%%%% 	KV-Diagramme	%%%%%%%%%%%%%%%%%%%%%%%%%%%%%%
%\usepackage{karnaugh}								% für KV-Diagramme
\usepackage{karnaugh-map}							% für KV-Diagramme
\usepackage{kvmap}									% für KV-Diagramme

%%%%%%%%%%%%%%%%%%%%%%% 	Codeeingabe		%%%%%%%%%%%%%%%%%%%%%%%%%%%%%%
\usepackage{minted}									% für Codehighlighting (z.B. VHDL)

%%%%%%%%%%%%%%%%%%%%%%% Quellenverzeichnis	%%%%%%%%%%%%%%%%%%%%%%%%%%%%%%
%\usepackage[style=science,backend=bibtex]{biblatex} % Für Quellenverzeichnis mit BibLatex
%\usepackage{filecontents}							% für festlegen von Quellen in der Tex Datei
%\addbibresource{NameDerDatei.bib}					% bindet die Quellen ein die vor documentclass 
% angelegt wurden

%%%%%%%%%%%%%%%%%%%%%%% 	für Befehle		%%%%%%%%%%%%%%%%%%%%%%%%%%%%%%
\usepackage{xparse}									% für \newcommand mit zwei optionalen Argumenten

%%%%%%%%%%%%%%%%%%%%%%% 	neue Pakete		%%%%%%%%%%%%%%%%%%%%%%%%%%%%%%


%%%%%%%%%%%%%%%%%%%%%%% 	eigene Befehle	%%%%%%%%%%%%%%%%%%%%%%%%%%%%%%%%%%%%%%%%%%%%%%%%%%%%%%%%%%%%%%%
%%%%%%%%%%%%%%%%%%%%%%% 	Laplace Symbol	%%%%%%%%%%%%%%%%%%%%%%%%%%%%%%
\newcommand{\vlaplace}{\mbox{\setlength{\unitlength}{0.1em}		% legt das Laplace Symbol vertikal als Bild
		\begin{picture}(20,10)									% an
			\put(1,-5){\circle*{4}}								% Laplace Seite unten
			\put(1,-3){\line(0,1){13}}
			\put(1,11){\circle{4}}
		\end{picture}							
	}
}
\newcommand{\vLaplace}{\mbox{\setlength{\unitlength}{0.1em}		% legt das Laplace Symbol vertikal als Bild
		\begin{picture}(20,10)									% an
			\put(1,-5){\circle{4}}								% Laplace Seite oben
			\put(1,-3){\line(0,1){13}}
			\put(1,11){\circle*{4}}
		\end{picture}							
	}
}
%%%%%%%%%%%%%%%%%%%%%%% 	Subsection		%%%%%%%%%%%%%%%%%%%%%%%%%%%%%%
\makeatletter													% um Probleme mit @ zu vermeiden (start)
\newcommand*{\addsubsec}{\secdef\@addsubsec\@saddsubsec}		% legt Makro addsubsec an
\newcommand*{\@addsubsec}{}										% schaut ob addsubsec noch nicht existiert
\def\@addsubsec[#1]#2{\subsection*{#2}\addcontentsline{toc}{subsection}{#1}		% definiert addsubsec
	%	\if@twoside\ifx\@mkboth\markboth\markright{#1}\fi\fi	% für wenn subsections im
	% Header stehen sollen
}
% \newcommand*{\@saddsubsec}[1]{\subsection*{#1}\@mkboth{}{}}	% für wenn subsections im
% Header stehen sollen
\makeatother													% um Probleme mit @ zu vermeiden (ende)

%%%%%%%%%%%%%%%%%%%%%%% 	Subsubsection	%%%%%%%%%%%%%%%%%%%%%%%%%%%%%%
\makeatletter													% um Probleme mit @ zu vermeiden (start)
\newcommand*{\addsubsubsec}{\secdef\@addsubsubsec\@saddsubsubsec}		% legt Makro addsubsubsec an
\newcommand*{\@addsubsubsec}{}									% schaut ob addsubsubsec noch nicht existiert
\def\@addsubsubsec[#1]#2{\subsubsection*{#2}\addcontentsline{toc}{subsubsection}{#1}		
	% definiert addsubsubsec
	%	\if@twoside\ifx\@mkboth\markboth\markright{#1}\fi\fi	% für wenn subsections im
	% Header stehen sollen
}
% \newcommand*{\@saddsubsec}[1]{\subsection*{#1}\@mkboth{}{}}	% für wenn subsections im
% Header stehen sollen
\makeatother													% um Probleme mit @ zu vermeiden (ende)

%%%%%%%%%%%%%%%%%%%%%%% Paragraph mit Linebreak	%%%%%%%%%%%%%%%%%%%%%%%%%%%%%%
%\makeatletter
%\renewcommand\paragraph{\@startsection{paragraph}{4}{\z@}%
%	{-2.5ex\@plus -1ex \@minus -.25ex}%
%	{1.25ex \@plus .25ex}%
%	{\normalfont\normalsize\bfseries}}
%\makeatother

\makeatletter													% um Probleme mit @ zu vermeiden (start)
\newcommand*{\addparagraphsec}{\secdef\@addparagraphsec\@saddparagraphsec}		% legt Makro addsubsubsec an
\newcommand*{\@addparagraphsec}{}								% schaut ob addsubsubsec noch nicht existiert
\def\@addparagraphsec[#1]#2{\paragraph*{#2}\addcontentsline{toc}{paragraph}{#1}%\mbox{}\\\\		
	% definiert addsubsubsec
	%	\if@twoside\ifx\@mkboth\markboth\markright{#1}\fi\fi	% für wenn subsections im
	% Header stehen sollen
}
% \newcommand*{\@saddsubsec}[1]{\subsection*{#1}\@mkboth{}{}}	% für wenn subsections im
% Header stehen sollen
\makeatother													% um Probleme mit @ zu vermeiden (ende)

\makeatletter
\newcommand*{\addparagraph}{\secdef\@addparagraph\@saddparagraph}
\newcommand*{\@addparagraph}{}
\def\@addparagraph[#1]#2{\addparagraphsec{#2}\mbox{}\\\\}
%\newcommand{\addparagraph}[1]{\addparagraphsec{#1}\mbox{}\\}
\makeatother

%%%%%%%%%%%%%%%%%%%%%%% 	Integral d		%%%%%%%%%%%%%%%%%%%%%%%%%%%%%%
\newcommand*\diff{\mathop{}\!\mathrm{d}}						% für das d beim Integral
\newcommand*\Diff[1]{\mathop{}\!\mathrm{d^#1}}					% für das d^(x) beim Integral

%%%%%%%%%%%%%%%%%%%%%%% senkrechtes Gleich	%%%%%%%%%%%%%%%%%%%%%%%%%%%%%%
\newcommand{\verteq}{\rotatebox{90}{$\,=$}}
\newcommand{\equalover}[2]{\overset{\scriptstyle\underset{\mkern4mu\verteq}{#2}}{#1}}

%%%%%%%%%%%%%%%%%%%%%%% "Entspricht"-Zeichen %%%%%%%%%%%%%%%%%%%%%%%%%%%%%
\newcommand{\eqvalent}[1]{\accentset{\wedge}{#1}}

%%%%%%%%%%%%%%%%%%%%%%% Tabular zweireihig  %%%%%%%%%%%%%%%%%%%%%%%%%%%%%%		
\NewDocumentCommand{\specialcell}{O{c} O{c} m m}{				% Zeilenumbruch in tabular-Umgebung
	\begin{tabular}[#1]{#2} {#3} \\ {#4}\end{tabular}}			% 1. [] t,c oder b 2. [] l,c oder r
%\newcommand{\specialcell}[4][c]{ 								% gleich wie oben, nur schlechter
	%	\begin{tabular}[#1]{@{}#2@{}}#3 \\ #4\end{tabular}}			% 

%%%%%%%%%%%%%%%%%%%%%%% 	neues Makro		%%%%%%%%%%%%%%%%%%%%%%%%%%%%%%

%%%%%%%%%%%%%%%%%%%%%%%%%%%%%%%%%%%%%%%%%%%%%%%%%%%%%%%%%%%%%%%%%%%%%%%%%%%%%%%%%%%%%%%%%%%%%%%%%%%%%%%%%%%
%%%%%%%%%%%%%%%%%%%%%%% 	Links			%%%%%%%%%%%%%%%%%%%%%%%%%%%%%%
\usepackage[hidelinks]{hyperref}					% Für Links auf Abschnitte usw ([] versteckt die Links)

%%%%%%%%%%%%%%%%%%%%%%%%%%%%%%%%%%%%%%%%%%%%%%%%%%%%%%%%%%%%%%%%%%%%%%%%%%%%%%%%%%%%%%%%%%%%%%%%%%%%%%%%%%%
%%%%%%%%%%%%%%%%%%%%%%% 	Quellen			%%%%%%%%%%%%%%%%%%%%%%%%%%%%%%
%\begin{filecontents}{NameDerDatei.bib}
%	@Book{NameDesFunktionsaufrufs,
	%		keywords = {},
	%		hyphenation = {},
	%		author = {},
	%		editor = {},
	%		translator = {},
	%		indextitle = {},
	%		title ={},
	%		shorttitle = {}}
%
%	@online{NameDesFunktionsaufrufs,
	%		label = {},
	%		title = {},
	%		year = {},
	%		url	= {},
	%		urldate = {}}
%\end{filecontents}